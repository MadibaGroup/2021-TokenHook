% !TEX root = ../main.tex

\section{Auditing Tools and \erc}\label{sec:tools}
Finally, we conducted an experiment on code auditing tools using the Solidity implementation of \sys to understand the current state of automated volunerabiliy testing. Our results illuminate the (in)completeness and error-rate of such tools on one specific use-case (related work studies, in greater width and less depth, a variety of use-cases~\cite{AuditTools}). We did not adapt older tools that support significantly lower versions of the Solidity compiler (\eg Oyente). We concentrated on Solidity as Vyper analysis is currently a paid services or penciled in for future support (\eg Slither). The provided version number is based on the GitHub repository; tools without a version are web-based and were used in 2020:

\begin{enumerate}[\setlength{\labelsep}{3pt}\setlength{\IEEElabelindent}{0pt}]
	\item EY Smart Contract \& Token Review by Ernst \& Young Global Limited~\cite{EYTool}.
	\item SmartCheck by SmartDec~\cite{SMARTCHECK}.
	\item Securify v2.0 by ChainSecurity~\cite{SECURIFYGIT,SECURIFY}.
	\item ContractGuard by GuardStrike~\cite{ContractGuard}.
	\item MythX by ConsenSys~\cite{MythX}.
	\item Slither Analyzer v0.6.12 by Crytic~\cite{SlitherDoc}.
	\item Odin by Sooho~\cite{Odin}.
\end{enumerate}

\subsection{Analysis of audit results}
A total of 82 audits have been conducted by these auditing tools that are summarized in Appendix \ref{subsec:audits}. Audits include best practices and security vulnerabilities. To compile the list of 82, we referenced the knowledge-base of each tool~\cite{SECURIFYGIT,SMARTCHECK,MythX,ContractGuard,SlitherDoc}, understood each threat, manually mapped the audit to the corresponding SWC registry~\cite{SWC}, and manually determined when different tools were testing for the same vulnerability or best practice (which was not always clear from the tools' own descriptions). Since each tool employs different methodology to analyze smart contracts (\eg comparing with violation patterns, applying a set of rules, using static analysis, \etc), there are false positives to manually check. Many false positives are not simply due to old/unmaintained rules but actually require tool improvement. We provide some examples in this section.

 \textit{MythX} detects \textit{Re-entrancy attack} in the \textit{noReentrancy} modifier. In Solidity, modifiers are not like functions. They are used to add features or apply some restriction on functions~\cite{SolidityModifer}. Using modifiers is a known technique to implement Mutex and mitigate re-entrancy attack~\cite{ReentrancyGuard}. This is a false positive and note that other tools have not identified the attack in modifiers.

\textit{ContractGuard} flags \textit{Re-entrancy attack} in \texttt{transfer()} function while countermeasures (based on both CEI and Mutex~\ref{subsec:ent}) are implemented.

\textit{Slither} detects two \textit{low level call} vulnerabilities~\cite{SlitherSetup}. This is due to use of \texttt{call.value()} that is recommend way of transferring ETH after \textit{Istanbul} hard-fork (EIP-1884).	Therefore, adapting analyzers to new standards can improve accuracy of the security checks.

\textit{SmartCheck} recommends not using \texttt{SafeMath} and check explicitly where overflows might be occurred. We consider this failed audit as false possible whereas utilizing \texttt{SafeMath} is a known technique to mitigate over/under flows. It also flags \textit{using a private modifier} as a vulnerability by mentioning, ``miners have access to all contracts' data and developers must account for the lack of privacy in Ethereum''. However private visibility in Solidity concerns object-oriented inheritance not confidentiality. For actual confidentiality, the best practice is to encrypt private data or store them off-chain. The tool also warns against \texttt{approve()} in \erc due to \textit{front-running attacks}. Despite EIP-1884, it still recommends using of \texttt{transfer()} method with stipend of 2300 gas. There are other false positives such as SWC-105 and SWC-112 that are passed by other tools.

\textit{Securify} detects the \textit{Re-entrancy} attack due to unrestricted writes in the \texttt{noReentrancy} modifier~\cite{SECURIFY}. Modifiers are the recommended approach and are not accessible by users. It also flags \textit{Delegatecall to Untrusted Callee (SWC-112)} while there is no usage of \texttt{delegatecall()} in the code. It might be due to use of \texttt{SafeMath} library which is an embedded library. In Solidity, embedded libraries are called by JUMP commands instead of \texttt{delegatecall()}. Therefore, excluding embedded libraries from this check might improve accuracy of the tool. Similar to \textit{SmartCheck}, it still recommends to use the \texttt{transfer()} method instead of \texttt{call.value()}.

\textit{EY token review} considers \texttt{decreaseAllowance} and \texttt{increaseAllowance} as standard \erc functions and if not implemented, recognizes the code as vulnerable to a \textit{front-running}. These two functions are not defined in the \erc standard~\cite{ERC20Std} and considered only by this tool as mandatory functions. There are other methods to prevent the attack while adhering \erc specifications (see Rahimian \etal for a full paper on this attack and the basis of the mitigation in \sys~\cite{ERC20MWA}). The tool also falsely detects the \textit{Overflow}, mitigated through \texttt{SafeMath}. Another identified issue is \textit{Funds can be held only by user-controlled wallets}. The tool warns against any token transfer to Ethereum addresses that belong to smart contracts. However, interacting with \erc token by other smart contracts was one of the main motivations of the standard. It also checks for maximum 50000 gas in \texttt{approve()} and 60000 in \texttt{transfer()} method. We could not find corresponding SWC registry or standard recommendation on these limitations and therefore consider them as informational.

\textit{Odin} raises \textit{Outdated compiler version} issue due to locking solidity version to 0.5.11. We have used this version due to its compatibility with other auditing tools. 

%Other tools cannot reason adequately when \texttt{Modifiers} are used.

%(1) EY Review Tool by Ernst \& Young Global Limited~\cite{EYTool}, (2) SmartCheck by SmartDec~\cite{SMARTCHECK}, (3) Securify v2.0 by ChainSecurity~\cite{SECURIFY}, (4) ContractGuard by GuardStrike~\cite{ContractGuard}, (5) MythX by ConsenSys~\cite{MythX}, (6) Slither Analyzer by Crytic~\cite{SlitherDoc} and (7) Odin by Sooho~\cite{Odin}.
%(this is more applicable to smart contracts than \erc tokens)

%