% !TEX root = ../main.tex

\section{Introduction}\label{sect:introduction}
The Ethereum blockchain~\cite{EthGit,EIP150} allows users to build and deploy decentralized applications (DApps) that can accept and use its protocol-level cryptocurrency ETH. Many DApps also issue or use custom tokens. Such tokens could be financial products, in-house currencies, voting rights for DApp governance, or other valuable assets. To encourage interoperability with other DApps and web applications (exchanges, wallets, \etc), the Ethereum community accepted a popular token standard (specifically for fungible tokens) called \erc~\cite{ERC20Std}. While numerous \erc extensions or replacements have been proposed, \erc remains prominent. Of the 2.5M~\cite{Alethio} smart contracts on the Ethereum network, 260K are tokens~\cite{TokenTracker} and 98\% of these tokens are \erc~\cite{EtherScan}. 

The development of smart contracts has been proven to be error-prone, and as a result, smart contracts are often riddled with security vulnerabilities. An early study in 2016 found that 45\% of smart contracts at that time had vulnerabilities~\cite{MakSm}. \erc token are subset of smart contracts and security is particularly important given that many tokens have considerable market capitalization (\eg USDT, BNB, UNI, DAI, \etc). As tokens can be held by commercial firms, in addition to individuals, and firms need audited financial statements in certain circumstances, the correctness of the contract issuing the tokens is now in the purview of professional auditors. Later, we examine one static anaylsis tool from a `big-four' auditing firm.

\subsubsection*{Contributions} Ethereum has undergone numerous security attacks that have collectively caused more than US\$100M in financial losses~\cite{DAO1,PeckShield,PartiyMultiSig,MyEthWallet,ParityFirstHack,ParitySecondHack}. Although research has been done on smart contract vulnerabilities in the past~\cite{EthSecServ}, we focus specifically on \erc tokens. 

\begin{enumerate}[\setlength{\labelsep}{3pt}\setlength{\IEEElabelindent}{0pt}]
\item We study all known vulnerabilities and cross-check their relevance to \erc token contracts, systematizing a comprehensive set of 82 distinct vulnerabilities and best practices. 
\item While not strictly a research contribution, we believe that our newly acquired specialized domain knowledge should be put to use. Thus, we provide a new \erc implementation, \sys, that is open source and freely available in both Vyper and Solidity.
\item \sys is positioned to increase software diversity: currently, no Vyper \erc implementation is considered a reference implementation, and only one Solidity implementation is actively maintained (OpenZeppelin's~\cite{OpenZepplin}). Relative to this implementation, \sys has enhanced security properties and stronger compliance with best practices. 
\item Perhaps of independent interest, we report on differences between Vyper and Solidity when implementing the same contract. \item We use \sys as a benchmark implementation to explore the completeness and precision of seven auditing tools that are widely used in industry to detect security vulnerabilities. We conclude that while these tools are better than nothing, they do not replace the role of a security expert in developing and reviewing smart contract code.
\end{enumerate}

% JC: List of vulnerabilities and best practices to subject matter expert
% JC: Vyper and Solidity
% JC: Audit tools compared to subject matter expert
% JC: Evaluate each tool -> identifying false positives, positives
% JC: Tools be improved

%To have a basis for comparison, we repeat the same audit process on the top ten \erc tokens by market capitalization.

%\footnote{Note to reviewers: we debated if our paper is an SoK or not, but we are open to having it appear in either category.}

% While we focus on \erc, many \erc proposed replacements (\ie ERC-223~\cite{Ref20}, 667~\cite{Ref21}, 721~\cite{Ref22}, 777~\cite{Ref23}, 827~\cite{Ref24}, 1155~\cite{Ref25}, 1377~\cite{Ref26}) either fully subsume \erc functionality (\ie they extend the \erc interface) or they overlap considerably.

% Other attacks will not apply to \erc tokens (\eg Delegatecall to untrusted callee, DoS with Failed Call) and are out of the scope. We also consider the best practices for improving the efficiency of token contracts (\eg as \erc compliance, Emitting token transfer events, modifying token allowance). 