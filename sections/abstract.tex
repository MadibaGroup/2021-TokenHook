% !TEX root = ../main.tex

\begin{abstract}
\erc is the most prominent Ethereum standard for fungible tokens. Tokens implementing the \erc interface can interoperate with a large number of already deployed internet-based services and Ethereum-based smart contracts. In recent years, security vulnerabilities in \erc have received special attention due to their widespread use and increased value. We systemize these vulnerabilities and their applicability to \erc tokens, which has not been done before. Next, we use our domain expertise to provide a new implementation of the \erc interface that is freely available in Vyper and Solidity, and has enhanced security properties and stronger compliance with best practices compared to the sole surviving reference implementation (from OpenZeppelin) in the \erc specification. Finally, we use our implementation to study the effectiveness of seven static analysis tools, designed for general smart contracts, for identifying ERC-20 specific vulnerabilities. We find large inconsistencies across the tools and a high number of false positives which shows there is room for further improvement of these tools. 
\end{abstract}

\begin{IEEEkeywords}
	
	Ethereum; Vyper; Solidity; \erc tokens; Security; Blockchain;
	
\end{IEEEkeywords}


%into a set of \num distinct vulnerabilities and best practices
%Reference \erc implementations have been slowly abandoned over time.