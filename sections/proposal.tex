% !TEX root = ../main.tex

%\begin{figure}[t!]
%	\centering
%	\includegraphics[width=1.0\linewidth]{figures/blockchain.png}
%	\caption{Architecture of the Ethereum blockchain in layers, including the interactive environment (\ie application layer). \erc tokens falls under the \textit{Smart Contracts} category in \textit{Contract Layer}.}\label{fig:blockchain}
%\end{figure}

\section{\sys}\label{sec:proposal}
\sys is our ERC20-compliant implementation written in Vyper (v. 0.2.8) and Solidity (v. 0.7.1). \sys is open source and available on Etherscan.\footnote{\sys deployed on Rinkeby at \url{https://bit.ly/33wDENx} (Solidity 0.7.1) and \url{https://bit.ly/3dXaaPc} (Vyper 0.2.8)}. It can be customized by developers, who can refer to each mitigation technique separately and address specific attacks. Required comments have been also added to clarify the usage of each function. Standard functionalities of the token (\ie \texttt{approve()}, \texttt{transfer()}, \texttt{transferFrom()}, \etc) have been unit tested. A demonstration of token interactions and event triggering can also be seen on Etherscan.\footnote{Etherscan: \url{https://bit.ly/33xHfL2}, \url{https://bit.ly/35TimMW} and \url{https://bit.ly/3eFAnAZ}}

% JC: This doesn't add much now that a more current version is on Rinkby: Mainnet at \url{https://bit.ly/35FMbAf} (Solidity 0.5.11)

%Among the layers of the Ethereum blockchain, \erc tokens fall under the \textit{Contract layer} in which DApps are executed. The presence of security vulnerability in supplementary layers affect the entire Ethereum blockchain, not necessarily \erc tokens. Therefore, vulnerabilities in other layers are assumed to be out of the scope (\eg \textit{Indistinguishable chains} at the data layer, the \textit{51\% attack} at the consensus layer, \textit{Unlimited nodes creation} at network layer, and \textit{Web3.js Arbitrary File Write} at application layer). Moreover, we exclude vulnerabilities identified in now outdated compiler versions~\cite{SWC}. (\eg SWC-109, SWC-111, SWC-118, SWC-129, \etc)

%Examples: \textit{Constructor name ambiguity} in versions before 0.4.22, \textit{Uninitialized storage pointer} in versions before 0.5.0, \textit{Function default visibility} in versions before 0.5.0, \textit{Typographical error} in versions before 0.5.8, \textit{Deprecated solidity functions} in versions before 0.4.25, \textit{Assert Violation} in versions before 0.4.10, \textit{Under-priced DoS attack} before EIP-150 \& EIP-1884)


\subsection{Security features}
In our research, we developed 82 security vulnerabilities in and best practices for \erc (see Appendix~\ref{subsec:audits}).  We highlight a few higher profile security vulnerabilities, typically ones that have been exploited in real world \erc tokens~\cite{SolidtySecBlog,EthSecServ,SoliditySecCon,ConsensysSecCon,LandoKL}. We provide more technical explanation of each in the appendix, and concentrate here on how \sys mitigates these attacks. While many of these attacks are no doubt very familiar to the reader, our emphasis in on their relevance to ERC-20.

\subsubsection{Multiple Withdrawal Attack (Background in Appendix~\ref{subsec:mwa})}
Without our counter-measure, an attacker can use a front-running attack~\cite{OrderingAttack,eskandari2019sok} to transfer more tokens than what is intended (approved) by the token holder. We secure the \texttt{transferFrom()} function by tracking transferred tokens to mitigate the \textit{multiple withdrawal} attack~\cite{ERC20MWA}. Securing \texttt{transferFrom()} function is fully compliant with the \erc standard without the need of introducing new functions such as \texttt{decreaseApproval()} and \texttt{increaseApproval()}. 

\subsubsection{Arithmetic Over/Under Flows (\ref{subsec:ovf})} 
In Solidity implementation, we use the \texttt{SafeMath} library in all arithmetic operations to catch over/under flows. Using it in Vyper is not required due to built-in integer overflow checks.

\subsubsection{Re-entrancy (\ref{subsec:ent})} 
At first glance, re-entrancy might seem inapplicable to \erc. However any function that changes internal state, such as balances, need to be checked. We use Checks-Effects-Interactions pattern (CEI)~\cite{CEI} in both Vyper and Solidity implementations to mitigate \textit{same-function re-entrancy} attack. Mutual exclusion (Mutex)~\cite{WiKiMutex} is also used to address \textit{cross-function re-entrancy} attack. Vyper supports Mutex by adding \texttt{@nonreentrant(<key>)} decorator on a function and we use \texttt{noReentrancy} modifier in Solidity to apply Mutex. Therefore, both re-entrancy variants are addressed in \sys. 
 
\subsubsection{Unchecked return values (\ref{subsec:urv})}
Unlike built-in support in Vyper, we must check the return value of \texttt{call.value()} in Solidity to revert failed fund transfers. It mitigates the \textit{unchecked return values} attack while making the token contract compatible with EIP-1884~\cite{EIP1884}. 

\subsubsection{Frozen Ether (\ref{subsec:feth})}
We mitigate this issue by defining a \texttt{withdraw()} function that allows the owner to transfer all ETH out of the token contract. Otherwise, unexpected ETH forced onto the token contract (\eg from another contract running \texttt{selfdestruct}) will be stuck forever. 

\subsubsection{Unprotected Ether Withdrawal (\ref{subsec:uew})}
We enforce authentication before transferring any funds out of the contract to mitigate \textit{unprotected Ether withdrawal}. Explicit check is added to the Vyper code and \texttt{onlyOwner} modifier is used in Solidity implementation. It allows only owner to call \texttt{withdraw()} function and protects Ether withdrawals. 

\subsubsection{State variable manipulation (\ref{subsec:svm})}
In Solidity implementation, we use embedded \texttt{Library} code (for \texttt{SafeMath}) to avoid external calls and mitigate the \textit{state variable manipulation} attack. It also reduces gas costs since calling functions in embedded libraries requires less gas than external calls.
	
%\subsubsection{Function visibility} We carefully define the visibility of each function. Most of the functions are declared as \texttt{External} (\eg \texttt{Approve()}, \texttt{Transfer()}, \etc) per specifications of \erc standard. (\cf Appendix \ref{subsec:pvis})

\subsection{Best practices and enhancements}
We also take into account a number of best practices that have been accepted by the Ethereum community to proactively prevent known vulnerabilities~\cite{TokenBP}. Again, we highlight several of these while placing the background details in the appendix.

\subsubsection{Compliance with \erc (\ref{subsec:compl})}
We implement all \erc functions to make it fully compatible with the standard. Compliance is important for ensuring that other DApps and web apps (\ie crypto-wallets, crypto-exchanges, web services, \etc) compose with \sys as expected. 

\subsubsection{External visibility (\ref{subsec:external})}
To improve performance, we apply an \texttt{external} visibility (instead of \texttt{public} visibility in the standard) for interactive functions (\eg \texttt{approve()} and \texttt{transfer()}, \etc).  External functions can read arguments directly from non-persistent \texttt{calldata} instead of allocating persistent memory by the EVM. 

\subsubsection{Fail-Safe Mode (\ref{subsec:failsf})}
We implement a `cease trade' operation that will freeze the token in the case of new security threats or new legal requirements (\eg Liberty Reserve ~\cite{LibertyReserve} or TON cryptocurrency~\cite{TON}). To freeze all functionality of \sys, the owner (or multiple parties) can call the function \texttt{pause()} which sets a lock variable. All critical methods are either marked with a \texttt{notPaused} modifier (in Solidity) or explicit check (in Vyper), that will throw exceptions until functionality is restored using \texttt{unpause()}. 

\subsubsection{Firing events (\ref{subsec:evnts})}
We define nine extra events: \texttt{Buy}, \texttt{Sell}, \texttt{Received}, \texttt{Withdrawal}, \texttt{Pause}, \texttt{Change}, \texttt{ChangeOwner}, \texttt{Mint} and \texttt{Burn}. The name of each event indicates its function except \texttt{Change} event which logs any state variable updates. It can be used to watch for token inconsistent behavior (\eg via TokenScope~\cite{TokenScope}) and react accordingly. 
	
\subsubsection{Proxy contracts (\ref{subsec:prxy})}
We choose to make \sys non-upgradable so it can be audited, and upgrades will not introduce new vulnerabilities that did not exist at the time of the initial audit. 
	
\subsubsection{Other enhancements (\ref{subsec:miner}), \ref{subsec:rvt}, \& \ref{subsec:slfd}}
We also follow other best practices such as not using batch processing in \texttt{sell()} function to avoid \textit{DoS with unexpected revert} issue, not using miner controlled variable in conditional statements, and not using \texttt{SELFDESTRUCT}.

\subsection{Implementing in Vyper vs. Solidity}
Although Vyper offers less features than Solidity (\eg no class inheritance, modifiers, inline assembly, function/operator overloading, \etc~\cite{SolidityDoc}), the Vyper compiler includes built-in security checks. Table~\ref{tab:compare} provides a comparison between the two from the perspective of \sys (see~\cite{Vyper1} for a broader comparison on vulnerabilities). Security and performance are advantages of Vyper. However, Vyper may not be a preferred option for production (``Vyper is beta software, use with care''~\cite{VyperReadme}), most of the auditing tools only support Solidity,\footnote{Vyper support is recently added to some tools (\eg Crytic-compile, Manticore and Echidna). Slither integration is still in progress~\cite{Crytic}} and Solidity currently enjoys widespread  implementation, developer tools, and developer experience.

% !TEX root = ../main.tex

\begin{table*}[]
	\centering
	\def\arraystretch{1.1}% 
	\begin{tabular}{|l|l|c|c|l|}
		\hline
		\multicolumn{2}{|c|}{\multirow{2}{*}{\textbf{\begin{tabular}[c]{@{}c@{}}Vulnerability (Vul.) or \\ Best Practice (BP.)\end{tabular}}}} & \multicolumn{2}{c|}{\textbf{\begin{tabular}[c]{@{}c@{}}\sys\\ Implementation\end{tabular}}} & \multicolumn{1}{c|}{\multirow{2}{*}{\textbf{Comment}}} \\ \cline{3-4}
		\multicolumn{2}{|c|}{} & \textbf{Vyper} & \textbf{Solidity} & \multicolumn{1}{c|}{} \\ \hline
		Arithmetic Over/Under Flows & Vul. & + &  & \begin{tabular}[c]{@{}l@{}}- Vyper includes built-in checks for over/under flows.\\ - \texttt{SafeMath} library is required in Solidity to mitigate the attack.\end{tabular} \\ \hline
		Re-Entrancy & Vul. & + &  & \begin{tabular}[c]{@{}l@{}}- \texttt{@nonreentrant} decorator places a lock on functions to mitigate the attack.\\ - \texttt{noReentrancy} modifier is required in Solidity.\end{tabular} \\ \hline
		Unchecked return values & Vul. & + &  & \begin{tabular}[c]{@{}l@{}}- It is already addressed in Vyper.\\ - There is a need in Solidity to check return values explicitly.\end{tabular} \\ \hline
		Code readability & BP. & + &  & \begin{tabular}[c]{@{}l@{}}- No inheritance in Vyper enforces simpler design.\\ - Solidity allows inline assemblies which is riskier and decreases readability.\end{tabular} \\ \hline
		Contract complexity & BP. & + &  & - 300 lines in Vyper have the same functionality as the Solidity with 500 lines. \\ \hline
		Auditable & BP. &  & + & - Most of the auditing tools are able to analyze Solidity contracts. \\ \hline
		Compatibility & BP. &  & + & \begin{tabular}[c]{@{}l@{}}- Majority of the current Ethereum projects are based on Solidity.\\ - Developers are more familiar with Solidity than Vyper.\end{tabular} \\ \hline
		Production readiness & BP. &  & + & \begin{tabular}[c]{@{}l@{}}- Vyper is not as mature as Solidity in terms of stability, documentation, etc.\\ - Solidity is adapted by a larger development community.\end{tabular} \\ \hline
	\end{tabular}
	\caption{Comparison of \sys implementation in Vyper and Solidity. The plus sign can be considered as an advantage. However, both versions offer the same level of security.}
	\label{tab:compare}
\end{table*}
	
\subsection{Need for another reference implementation}
The authors of the \erc standard reference two sample Solidity implementations: one that is actively maintained by OpenZeppelin~\cite{OpenZepplin} and one that has been deprecated by ConsenSys~\cite{ConsensysToken} (and now refers to the OpenZeppelin implementation). As expected, the OpenZeppelin template is very popular within the Solidity developers~\cite{OPZ1,OPZ2,OPZ3}.

OpenZeppelin's implementation is actually part of a small portfolio of implementations (ERC20, ERC721, ERC777, and ERC1155). Code reuse across the four implementations adds complexity for a developer that only wants \erc. This might be the reason for not supporting Vyper in OpenZeppelin's implementation. No inheritance in Vyper requires different implementation than the current object-oriented OpenZeppelin contracts. Further, most audit tools are not able to import libraries/interfaces from external files (\eg SafeMath.sol, IERC20.sol). By contrast, \sys uses a flat layout in a single file that is specific to \erc. It does not use inheritance in Solidity which allows similar implementation in Vyper.  

%In terms of best practices, OpenZeppelin requires other optimizations such as, locking the \textit{pragma} to avoid using outdated compiler versions by developers (\cf SWC-103), emit a \textit{Change} event (\cf TokenScope~\cite{TokenScope}) when updating state variables (\eg \texttt{\_decimals=decimals\_} in \texttt{\_setupDecimals()}). Initializing \textit{totalSupply} in the constructor is also recommended. As we implemented, using \texttt{external} visibility instead of public increases readability and consumes less gas. Finally, these are very minor but it helps to have more readable token contract by avoiding similar variable names (\eg \texttt{\_name=name\_} in \texttt{constructor()}) and using \textit{mixedCase} format when declaring variable and functions.

\sys makes other improvements over the OpenZeppelin implementation. For example, OpenZeppelin introduces two new functions to mitigate the multiple withdraw attack: \texttt{increaseAllowance()} and \texttt{decreaseAllowance()}. However these are not part of the \erc standard and are not  interoperable with other applications that expect to use \texttt{approve()} and \texttt{transferFrom()}. \sys secures \texttt{transferFrom()} to prevent the attack (following~\cite{ERC20MWA}) and is interoperable with legacy DApps and web apps. Additionally, \sys mitigates the \textit{frozen Ether} issue by introducing a \texttt{withdraw()} function, while ETH forced into the OpenZeppelin implementation is forever unrecoverable. Both contracts implement a \textit{fail-safe mode}, however this logic is internal to \sys, while OpenZeppelin requires an external \texttt{Pausable.sol} contract.

Diversity in software is important for robustness and security~\cite{FSA97,FHS97}. For \erc, a variety of implementations will reduce the impact of a single bug in a single implementation. For example, between 17 March 2017 and 13 July 2017, OpenZeppelin's implementation used the wrong interface and affected 130 tokens~\cite{ErcBug}. \sys increases the diversity of \erc Solidity implementations and addresses the lack of a reference implementation in Vyper.

% JC: Best guess at this paragraph (I could tell which were statements about us and which were statements about OpenZeppplin)

%These are only useful for tokens with a fixed exchange rate, which is managed by the \texttt{exchangeRate} variable.

%\begin{enumerate}
%	\item \textbf{Buying tokens:} \erc tokens can be offered to users for purchase. Users call the \texttt{buy()} function which accepts ETH (\ie defined as \textit{payable}) to be held by the \erc contract. The contract calculates the equivalent number of tokens based on the current exchange rate, increases the token balance of the buyer, and logs a \texttt{Buy} event.
%	\item \textbf{Selling tokens:} By using \texttt{sell()} function, token holders can send back tokens to the contract and receive ETH in return as long as the contract holds ETH (see withdrawing ETH below). After each exchange, a \texttt{Sell} event triggers. 
%	\item \textbf{Withdrawing Ether:} This function can be called only by the contract owner (or with a set of authorizations in a multi-owner implementations). The \texttt{withdraw()} function will transfer ETH out of the contract. This mitigates the unexpected ETH issue. Transferring ETH out of the contract logs a \texttt{Withdrawal} event.
%\end{enumerate}
%
%These extra features allow the purchase and sale of tokens independently of an exchange service for fixed priced tokens.

%would be a financial advantage for new tokens and makes them independent of crypto-exchanges. All the required functionalities are directly supported by the token contract and no additional external services are required. This feature is a financial advantage for new \erc tokens and reduces buyers doubts. They can return purchased token at any time and receive the equivalent in ETH. Another option for them is to wait for the token to be listed by crypto-exchanges (if it ever happens). Otherwise, they would not be able to exchange tokens if this feature is not support by the token contract.
