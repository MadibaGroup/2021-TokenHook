% !TEX root = ../main.tex

%\begin{figure}[t!]
%	\centering
%	\includegraphics[width=1.0\linewidth]{figures/blockchain.png}
%	\caption{Architecture of the Ethereum blockchain in layers, including the interactive environment (\ie application layer). \erc tokens falls under the \textit{Smart Contracts} category in \textit{Contract Layer}.}\label{fig:blockchain}
%\end{figure}

\section{\sys}\label{sec:proposal}
\sys, our ERC20-compliant implementation written in Solidity. {\blue It covers all functionalities supported by existing \erc contracts while mitigates more security vulnerabilities.} \sys is open source and available on Etherscan, where it has been tested with MetaMask and deployed on Mainnet\footnote{Etherscan: \url{https://bit.ly/35FMbAf}}. It can be customized by developers, who can refer to each mitigation technique separately and address a specific attack. Required comments have been also added to clarify usage of each part. Standard functionalities of the token (\ie \texttt{approve()}, \texttt{transfer()}, \etc) have been unit tested. A demonstration of token interactions and event triggering can also be seen on Etherscan\footnote{Etherscan: \url{https://bit.ly/33xHfL2}, \url{https://bit.ly/35TimMW}}.

Among the layers of the Ethereum blockchain, \erc tokens fall under the \textit{Contract layer} in which DApps are executed. The presence of security vulnerability in supplementary layers affect the entire Ethereum blockchain, not necessarily \erc tokens. Therefore, vulnerabilities in other layers are assumed to be out of the scope (\eg \textit{Indistinguishable chains} at the data layer, the \textit{51\% attack} at the consensus layer, \textit{Unlimited nodes creation} at network layer, and \textit{Web3.js Arbitrary File Write} at application layer). Moreover, we exclude vulnerabilities identified in now outdated compiler versions, for example:
\begin{itemize}[noitemsep,topsep=0pt]
	\item \textit{Constructor name ambiguity} in versions before 0.4.22.
	\item \textit{Uninitialized storage pointer} in versions before 0.5.0.
	\item \textit{Function default visibility} in versions before 0.5.0
	\item \textit{Typographical error} in versions before 0.5.8.
	\item \textit{Deprecated solidity functions} in versions before 0.4.25.
	\item \textit{Assert Violation} in versions before 0.4.10.
	\item \textit{Under-priced DoS attack} before EIP-150 \& EIP-1884.
\end{itemize}

{\blue 
\subsection{Security features}
We examine general attack vectors and cross-check their applicability to \erc tokens (See section \ref{sec:vul}). We then consider mitigation techniques in our implementation as summarized below:
\begin{enumerate}[noitemsep,topsep=0pt]
	\item Securing \texttt{transferFrom()} function by tracking transferred tokens. It mitigates \textit{multiple withdrawal} attack wherein an attacker may transfer more token than the approved amount by the token holder. (\cf Section \ref{subsec:mwa})

	\item Using \texttt{SafeMath} library in all arithmetic operations to catch over/under flows. (\cf Section \ref{subsec:ovf})

	\item Implementing \texttt{noReentrancy} modifier for external functions. It enforces Mutex and mitigate \textit{same-function re-entrancy} and \textit{cross-function re-entrancy} attacks. (\cf Section \ref{subsec:ent})

	\item Checking the returned value of \texttt{call.value()} to revert failed fund transfers in \texttt{sell()} and \texttt{withdraw()} functions. It mitigates \textit{Unchecked return values} attack while making the token contract compatible with EIP-1884\cite{EIP1884}. (\cf Section \ref{subsec:urv})

	\item Mitigating \textit{Frozen Ether} issue by defining \texttt{withdraw()} function. It allows the owner to transfer ETH out of the token contract. Otherwise, sent ETH to the token will be stuck forever (\cf Section \ref{subsec:feth})

	\item Applying \texttt{onlyOwner} modifier to \texttt{withdraw()} function. It Mitigates \textit{Unprotected Ether Withdrawal} issue by enforcing authentication before transferring any funds out of the contract. (\cf Section \ref{subsec:uew})

	\item Adding \texttt{Library} code to the \erc contract's code (\ie Using embedded libraries). It mitigates \textit{State variable manipulation} attack and avoids updating internal variables by external contracts. Also, calling functions in embedded libraries requires less gas compared to invoking them via external calls (\cf Section \ref{subsec:svm})
	
	\item Avoiding default \texttt{Public} visibility by explicitly defining visibility of each function. Most of the functions are declared as \texttt{External} (\eg \texttt{Approve()}, \texttt{Transfer()}, \etc) per specifications of \erc standard. (\cf Section \ref{subsec:pvis})
\end{enumerate}

\subsection{Complementary features}
In addition to reviewing known vulnerabilities, we also took into account a number of best practices (See section \ref{sec:bp}). They improve the performance of \sys and complement its features:
\begin{enumerate}[noitemsep,topsep=0pt]
	\item Implementing all functions to make it fully compatible with the \erc standard. Therefore, there would not be any failed call for other DApps (\ie crypto-wallets, crypto-exchanges, web services, \etc) which are expecting them (Section \ref{subsec:compl})

	\item Enforcing \texttt{External} visibility for interactive functions (\eg \texttt{approve()} and \texttt{transfer()}, \etc) to improve performance. Functions declared as \texttt{External} can read arguments directly from non-persistent \texttt{calldata} area, instead of allocating persistent memory by EVM. (Section \ref{subsec:external})

	\item Ability to stop the token in case of new security threats or legal requirements (\eg Liberty Reserve \cite{LibertyReserve} or TON cryptocurrency\cite{TON}). To freeze all functionality of \sys, the owner (or multiple parties) can call \texttt{pause()} function. It then sets a lock variable and methods are marked with \texttt{notPaused} modifier, throw exceptions until functionality is restored using \texttt{unpause()}. (Section \ref{subsec:failsf})

	\item Defining nine extra events: \texttt{Buy}, \texttt{Sell}, \texttt{Received}, \texttt{Withdrawal}, \texttt{Pause}, \texttt{Change}, \texttt{ChangeOwner}, \texttt{Mint} and \texttt{Burn}. \texttt{Change} event logs any state variable updates that can be used to watch for token inconsistent behavior (\eg via TokenScope\cite{TokenScope}) and react accordingly. (Section \ref{subsec:evnts})
	
	\item Implementing \texttt{sell()} and \texttt{buy()} for exchanging tokens and ETH. \texttt{sell()} allows token holders to exchange tokens for ETH and \texttt{buy()} accepts ETH by adjusting buyer's token balance. {\blue This can be considered as a financial incentive in which it is possible to buy and sell tokens at a fixed price by the token contract (\eg Launching an Initial Coin Offering (ICO), Prediction market sell). Otherwise, buyers will have to wait for the token to be listed on crypto-exchanges (if it ever happens) or look for a buyer/seller themselves. In addition, it reduces the cost of token exchange by eliminating crypto-exchange's fees.}

	\item Making \sys as non-upgradable token for auditing purpose. Initial token audit might show it as secure while the upgraded versions contains new vulnerabilities that did not exist at the time of initial audit.
	
	\item Considering other best practices such as not using batch processing in \texttt{sell()} function to avoid \textit{DoS with Unexpected revert} issue, not using miner controlled variable in conditional statements and not using \texttt{SELFDESTRUCT}.
\end{enumerate}

\subsection{Increased diversity}
The authors of the \erc standard reference two sample implementations from OpenZeppelin\cite{OpenZepplin} and ConsenSys\cite{ConsensysToken}. ConsenSys implementation is deprecated (according to the GitHub page) and OpenZeppelin template is the most referenced implementation by the community\cite{OPZ1,OPZ2,OPZ3}. Having different \erc implementation provides more variety in implementations and minimizes the possibility of bugs that existed in the past. Between 17 March 2017 and 13 July 2017, OpenZeppelin implemented the wrong interface in their framework that affected 130 tokens\cite{ErcBug}. Diversity in \erc implementation can reduce the impact of such errors and increase the robustness of \erc tokens. Additionally, \sys has the following advantages:
\begin{itemize}[noitemsep,topsep=0pt]
	\item Unlike OpenZeppelin which introduces two new functions to mitigate \textit{front-running attack} (\ie \texttt{increaseAllowance()} and \texttt{decreaseAllowance()}), \sys secures standard \erc function (\ie \texttt{transferFrom()}). DApps can interact with standard \texttt{approve()} and \texttt{transferFrom()} methods without adapting their code to these new functions. \sys is therefore fully compliant with the \erc specification and can interact with already developed and legacy DApps.

	\item \sys mitigates \textit{Frozen Ether} issue by introducing \texttt{withdraw()} function while sent ETH to OpenZeppelin contract are unrecoverable.

	\item \textit{Fail-Safe Mode} is a built-in feature of \sys while OpenZeppelin requires incorporation of \texttt{Pausable.sol} contract.

	\item OpenZeppelin requires other optimizations such as \textit{Locking the pragma}, Emitting \textit{Change} event (\cf TokenScope\cite{TokenScope}) when updating state variables (\eg \texttt{\_decimals=decimals\_} in \texttt{\_setupDecimals()}), \textit{Initializing totalSupply in constructor}, using \texttt{External} visibility instead of Public to increase readability (\ie no internal call) and consume less gas, Avoiding similar variable names (\eg \texttt{\_name=name\_} in \texttt{constructor()}), Using \textit{mixedCase} format when declaring variable and functions (\eg \texttt{\_symbol}, \texttt{\_decimals}), \etc

	\item Using reusable codes has made the OpenZeppelin code complex and challenging for security tools. Developers need to manually check the code for vulnerabilities instead of using vulnerability assessment tools. Additionally, most of the security tools are not able to import libraries/interfaces from external files (\eg SafeMath.sol, IERC20.sol). \sys has a flat layout and all codes are in one file. It is easier for developers to understand and upload it to audit tools for vulnerability assessment.
\end{itemize}

}

%These are only useful for tokens with a fixed exchange rate, which is managed by the \texttt{exchangeRate} variable.

%\begin{enumerate}
%	\item \textbf{Buying tokens:} \erc tokens can be offered to users for purchase. Users call the \texttt{buy()} function which accepts ETH (\ie defined as \textit{payable}) to be held by the \erc contract. The contract calculates the equivalent number of tokens based on the current exchange rate, increases the token balance of the buyer, and logs a \texttt{Buy} event.
%	\item \textbf{Selling tokens:} By using \texttt{sell()} function, token holders can send back tokens to the contract and receive ETH in return as long as the contract holds ETH (see withdrawing ETH below). After each exchange, a \texttt{Sell} event triggers. 
%	\item \textbf{Withdrawing Ether:} This function can be called only by the contract owner (or with a set of authorizations in a multi-owner implementations). The \texttt{withdraw()} function will transfer ETH out of the contract. This mitigates the unexpected ETH issue. Transferring ETH out of the contract logs a \texttt{Withdrawal} event.
%\end{enumerate}
%
%These extra features allow the purchase and sale of tokens independently of an exchange service for fixed priced tokens.

%would be a financial advantage for new tokens and makes them independent of crypto-exchanges. All the required functionalities are directly supported by the token contract and no additional external services are required. This feature is a financial advantage for new \erc tokens and reduces buyers doubts. They can return purchased token at any time and receive the equivalent in ETH. Another option for them is to wait for the token to be listed by crypto-exchanges (if it ever happens). Otherwise, they would not be able to exchange tokens if this feature is not support by the token contract.
