% !TEX root = ../main.tex

\subsection{Comparing audits}
After manually overriding the false positives, the average percentage of passed checks for \sys reaches to \prct. To pass the one missing check and reach a 100\% success rate across all tools, we prepared the same code in Solidity version 0.8.0, however it cannot be audited anymore with most of the tools. 

We repeated the same auditing process on the top ten tokens based on their market cap\cite{EtherScan}. The result of all these evaluation have been summarized in Table\ref{tab:summary} by considering false positives as failed audits. This provide the same evaluation conditions across all tokens. Since each tool uses different analysis methods, number of occurrences are considered for comparisons. For example, MythX detects two \textit{re-entrancy} in \sys; therefore, two occurrences are counted instead of one. 

As it can be seen in Table \ref{tab:summary}, \sys has the least number of security flaws (occurrences) compared to other tokens. We stress that detected security issues for \sys are all false positives. {\blue Considering number of vulnerability might not be a perfect measurement, but it provides at least an approximation of the token security factor compared to other tokens. It should also be noted that the existence of vulnerability does not mean that it can be exploited \cite{VulExp}. It could have been the subject of a separate study that we leave it for future works}

% !TEX root = ../main.tex

\begin{table}[t!]
\centering

\begin{tabular}{|l|c|c|c|c|c|c|c|c|}
	\rcl
	\multicolumn{1}{|c|}{\ccl} & \multicolumn{7}{c|}{\ccl\tx{Auditing Tool}} & \ccl \\ \cline{2-8}
	\rcl 
	\multicolumn{1}{|c|}{\multirow{-2}{*}{\ccl\tx{\begin{tabular}[c]{@{}c@{}}ERC-20\\ Token\end{tabular}}}} & \begin{tabular}[c]{@{}c@{}}EY Token\\ Review\end{tabular} & \begin{tabular}[c]{@{}c@{}}Smart\\ Check\end{tabular} & Securify & \begin{tabular}[c]{@{}c@{}}MythX\\ (Mythril)\end{tabular} & \begin{tabular}[c]{@{}c@{}}Contract\\ Guard\end{tabular} & Slither & Odin & \multirow{-2}{*}{\ccl\tx{\begin{tabular}[c]{@{}c@{}}Total\\ issues\end{tabular}}} \\ \hline
	\tx{\sys} & 9 & 11 & 4 & 2 & 10 & 2 & 2 & \tx{40} \\ \hline
	\tx{TUSD} & 20 & 11 & 2 & 1 & 14 & 16 & 6 & \tx{70} \\ \hline
	\tx{PAX} & 16 & 9 & 6 & 4 & 16 & 13 & 9 & \tx{73} \\ \hline
	\tx{USDC} & 17 & 9 & 6 & 5 & 18 & 15 & 10 & \tx{80} \\ \hline
	\tx{INO} & 11 & 10 & 14 & 8 & 14 & 24 & 12 & \tx{93} \\ \hline
	\tx{HEDG} & 10 & 28 & 11 & 1 & 29 & 24 & 16 & \tx{119} \\ \hline
	\tx{BNB} & 13 & 21 & 12 & 13 & 41 & 39 & 3 & \tx{142} \\ \hline
	\tx{MKR} & 11 & 27 & 38 & 9 & 16 & 34 & 18 & \tx{153} \\ \hline
	\tx{LINK} & 12 & 27 & 38 & 9 & 16 & 34 & 18 & \tx{181} \\ \hline
	\tx{USDT} & 12 & 29 & 8 & 17 & 46 & 55 & 30 & \tx{197} \\ \hline
	\tx{LEO} & 32 & 25 & 8 & 23 & 70 & 75 & 19 & \tx{252} \\ \hline
\end{tabular}
\caption{Security flaws detected by seven auditing tools in \sys (the proposal) compared to top 10 \erc tokens by market capitalization in May 2020. \sys has the lowest reported security issues (occurrences). \label{tab:summary}}
\end{table}

