% !TEX root = ../main.tex

\section{Conclusion}

98\% of tokens on Ethereum today implement \erc. While attention has been paid to the security of Ethereum DApps, threats to tokens can be specific to \erc functionality. In this paper, we provide a detailed study of \erc security, collecting and deduplicating applicable vulnerabilities and best practices, examining the ability of seven audit tools. Most importantly, we provide a concrete implementation of \erc called \sys. It is designed to be secure against known vulnerabilities, and can serve as a second reference implementation to provide software diversity. We test it at Solidity version 0.5.11 (due to the limitation of the audit tools) and also provide it at version 0.8.4. Vyper implementation is also provided at version 0.2.8 to make \erc contracts more secure and easier to audit. \sys can be used as template\footnote{Compatible Solidity version of \sys (v. 0.5.11) deployed on Mainnet at \url{https://bit.ly/35FMbAf} and the latest Solidity (v. 0.8.4) on Rinkeby \url{https://bit.ly/3tI139S}. Vyper code at \url{https://bit.ly/3dXaaPc}.} to deploy new \erc tokens (\eg ICOs, DApps, etc), migrate current vulnerable deployments, and to benchmark the precision of Ethereum audit tools.


%Further, there is no vulnerability reference site (\cf the SWC Registry) specifically for \erc tokens.

%, and auditing 10 \erc deployments