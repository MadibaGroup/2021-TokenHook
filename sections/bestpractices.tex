% !TEX root = ../main.tex

\section{A sample of best practices}\label{section:bp}

In addition to reviewing known vulnerabilities, we also took into account a number of best practices for developing \erc on Ethereum. Again, due to space, we highlight a few that have been accepted by the Ethereum community to proactively prevent known vulnerabilities~\cite{TokenBP}. Some best practices are specific to \erc, while others are generic for all DApps---in which case, we discuss their relevance to \erc and to \sys.

\subsection{Compliance with \erc.}

According to the \erc specifications, all six methods and two events must be implemented and are not optional. Moreover, ignoring them can cause failed function calls by other applications (\ie crypto-wallets, crypto-exchanges, web services, \etc) which are expecting them. Tokens that do not implement all methods (\eg \texttt{approve()} or \texttt{transferFrom()}) might also be vulnerable to complex attacks (\eg Fake deposit vulnerability\cite{DEPOSafe}, Missing return value bug\cite{ErcBug}). For \sys, we implement all the required methods, and add some complementary functions such as \texttt{sell()} and \texttt{buy()}. \texttt{sell()} allows token holders to exchange tokens for ETH and \texttt{buy()} accepts ETH by adjusting buyer's token balance. {\blue This can be considered as a financial incentive in which it is possible to buy and sell tokens at a fixed price by the token contract. Otherwise, buyers will have to wait for the token to be listed on crypto-exchanges (if it ever happens) or look for a buyer themselves. In addition, it reduces the cost of token exchange by eliminating crypto-exchange's fees.}

%\subsection{Firing events.}
%
%In \erc standard, there are two defined events: \texttt{Approval} and \texttt{Transfer}. The first event logs successful allowance changes by token holders and the second logs successful token transfers by the \texttt{transfer()} and \texttt{transferFrom()} methods. These two events must be fired to notify external application on occurred changes. The external application might use them to update balances, show UI notifications, or to check new token approvals. In addition to the above logs, we define six extra events in \sys that are \texttt{Buy}, \texttt{Sell}, \texttt{Received}, \texttt{Withdrawal}, \texttt{Change} and \texttt{Pause}. These can be used to watch for token events and react accordingly.

%\subsection{Unexpected revert.}
%
%A function that attempts to complete many operations that individually may revert could deadlock if one operation always fails. For example, \texttt{transfer()} can throw an exception---if one transfer in a sequence fails, the whole sequence fails. One standard practice is to account for ETH owed and require withdrawals through a dedicated function. In \sys, ETH is only transferred to a single party in a single function \texttt{sell()}. It seems overkill to implement a whole accounting system for this. As a consequence, a seller that is incapable of receiving ETH (\eg operating from a contract that is not payable) will be unable to sell their tokens for ETH. However they can recover by transferring the tokens to a new address to sell from. 

\subsection{External visibility.}\label{subsec:external}

Solidity supports two types of \textit{function calls}: internal and external~\cite{SolidityDoc}. Internal function calls expect arguments to be in memory and the EVM copies the arguments to memory. Internal calls use \texttt{JUMP} opcodes instead of creating an \textit{EVM call}.\footnote{Also known as ``message call'' when a contract calls a function of another contract.} Conversely, External function calls create an \textit{EVM call} and can read arguments directly from the \texttt{calldata} space. This is cheaper than allocating new memory and designed as a read-only byte-addressable space where the data parameter of a transaction or call is held~\cite{EthInDepth}. A best practice is to use external functions when we expect that functions will be called externally. We follow this recommendation in \sys by using \texttt{External} visibility on all such methods instead of \texttt{Public}. 

\subsection{Fail-Safe Mode.}

In the case of a detected anomaly or attack on a deployed \erc token, the functionality of the token can be frozen pending further investigation. { \blue Similar to Liberty Reserve digital currency service~\cite{LibertyReserve}, governments may compel to stop the token's ability to transact. } To freeze all functionality, the owner of the token can call \texttt{pause()} function. It then sets a lock variable and methods are marked with \texttt{notPaused} modifier, throw exceptions until functionality is restored using \texttt{unpause()}. We apply \texttt{notPaused} modifier to all external functions.
%(\eg \texttt{transfer()}, \texttt{sell()}, \etc).

%\subsection{Global or Miner controlled variables.}
%
%Since malicious miners have the ability to manipulate global Solidity variables (\eg \texttt{block.timestamp}, \texttt{block.number}, \texttt{block.difficulty}, \etc), it is recommended to avoid these variables. \sys does not use such variables for conditional execution, authentication, or randomness.

%\subsection{Proxy contracts.}
%
%An \erc token can be deployed with a pair of contracts: a proxy contract that passes through all the function calls to a second functioning \erc contract~\cite{ProxyContract,ProxyPatterns}. One use of proxy contract is when upgrades are required---a new functional contract can be deployed and the proxy is modified to point at the update. We concentrate on building a secure, standalone implementation. Users of \sys can freely proxy \sys as they choose. 


% \subsubsection{DoS with block gas limit.}
% The use of loops in contracts is not efficient and requires considerable amount of Gas to execute. It might also cause DoS attack since blocks has a \textit{Gas limit}. If execution of a function exceeds the block gas limit, all transactions in that block will fail. Hence, we do not use loops and rely on \texttt{mappings} variables. They store data in collection of key value pairs and are more efficient.